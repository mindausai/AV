%
%  outline latex source document for AILP assignment 1.
%  use pdflatex to format this.
%
\documentclass[10pt,a4paper,twocolumn]{article}
\usepackage{amssymb,amsmath}            % if some maths is needed
\usepackage{graphicx}                   % if any images are to be included
% pick a different font if desired
\usepackage{times}

\title{AILP (2011) Assignment 1 Report}  % AILP: please use this title.
\author{The author}                      % replace with name or exam number
\date{10th Octover 2011}                 % replace with actual date

\begin{document}

\maketitle  % insert title, author info
%
\section{Introduction}

This report describes the work done for the first assignment in the AILP
course.  It gives the aims and hypothesis that guided the work;
describes the algorithms that were implemented;  reports the results
of experiments that were run;  and analyses these results.

\section{Aims and hypothesis}

The aim of the assignment is to pre-process the images for

The working hypothesis is as follows:
\begin{quote}
  The inclusion of some pre-processing steps in the character recognition
  process will improve the performance of the system.
\end{quote}1


\section{Algorithms and implementation}

For this purpose the following pre-processing algorithms were implemented:
\begin{enumerate}
\item Position normalisation;
\item Slant normalisation;
\item Scaling normalisation (bilinear interpolation);
\item Normalizing imiges using therhold.
\end{enumerate}

\subsection{Position normalisation}

The basic idea for position normlisation was to move digits centriod to a diffrent location, the center of image. To do that, I needed to calculate digits centroid using three formulas:
\begin{equation}
x_c=\frac{1}{S} \sum_{y=1}^H
\label{eq1}
\end{equation}

\subsection{Slant normalisation}

Another algorithm.

\section{Experiments and results}

\section{Discussion and Conclusion}

\subsection{Formatting: tables}

An example of a table is shown as Table \ref{table1}. Somewhat 
different styles are allowed according to the type and purpose of the 
table. 

\begin{table} [t,h]
\caption{\label{table1} \textit{This is an example of a table.}}
\vspace{2mm}
\centerline{
\begin{tabular}{|c|c|}
\hline
ratio & decibels \\
\hline  \hline
1/1 & 0 \\
2/1 & $\approx 6$ \\
3.16 & 10 \\
10/1 & 20 \\ 
1/10 & -20 \\
100/1 & 40 \\
1000/1 & 60 \\
\hline
\end{tabular}}
\end{table}

To include text without formatting, use this
(scriptsize uses a significantly smaller font,
intermediate sizes are footnotesize and small):
{\scriptsize
\begin{verbatim}
I\O     1     2     3     4     5     6     7     8     9    10
  1  71.2   8.8   1.2   0.0   2.5   3.8   7.5   0.0   5.0   0.0
  2   0.0  87.5   1.2   0.0   2.5   2.5   0.0   5.0   0.0   1.2
  3   0.0   0.0  67.5   5.0   1.2  11.2   3.8   7.5   3.8   0.0
  4   0.0   0.0   1.2  62.5   3.8  22.5   0.0   6.2   2.5   1.2
  5   0.0   2.5   0.0   0.0  76.2   0.0   1.2   6.2   0.0  13.8
  6   5.0   1.2   6.2  21.2   5.0  47.5   1.2   5.0   1.2   6.2
  7  17.5   6.2   3.8   0.0   5.0   0.0  57.5   0.0  10.0   0.0
  8   0.0   0.0   2.5   1.2   8.8   0.0   0.0  73.8   2.5  11.2
  9  11.2   0.0   2.5   8.8   2.5   3.8   5.0   2.5  61.3   2.5
 10   1.2   0.0   0.0   2.5  20.0   0.0   0.0  12.5   0.0  63.7
\end{verbatim}
}

If you want to use both columns, put it in a figure*:
(figure* uses both columns, figure just 1):
it is likely to float away to an unexpecte place, though.
\begin{figure*}[h]
  \centering
{\small
\begin{verbatim}
I\O     1     2     3     4     5     6     7     8     9    10
  1  71.2   8.8   1.2   0.0   2.5   3.8   7.5   0.0   5.0   0.0
  2   0.0  87.5   1.2   0.0   2.5   2.5   0.0   5.0   0.0   1.2
  3   0.0   0.0  67.5   5.0   1.2  11.2   3.8   7.5   3.8   0.0
  4   0.0   0.0   1.2  62.5   3.8  22.5   0.0   6.2   2.5   1.2
  5   0.0   2.5   0.0   0.0  76.2   0.0   1.2   6.2   0.0  13.8
  6   5.0   1.2   6.2  21.2   5.0  47.5   1.2   5.0   1.2   6.2
  7  17.5   6.2   3.8   0.0   5.0   0.0  57.5   0.0  10.0   0.0
  8   0.0   0.0   2.5   1.2   8.8   0.0   0.0  73.8   2.5  11.2
  9  11.2   0.0   2.5   8.8   2.5   3.8   5.0   2.5  61.3   2.5
 10   1.2   0.0   0.0   2.5  20.0   0.0   0.0  12.5   0.0  63.7
\end{verbatim}
}

  \caption{Confusion Matrix}
  
\end{figure*}

\subsection{Maths, if needed}

%
%\vspace{-3mm}
\begin{equation}
x(t) = s(f_\omega(t))
\label{eq1}
\end{equation}
where \(f_\omega(t)\) is a special warping function
\begin{equation}
f_\omega(t)=\frac{1}{2\pi j}\oint_C \frac{\nu^{-1k}d\nu}
{(1-\beta\nu^{-1})(\nu^{-1}-\beta)}
\label{eq2}
\end{equation}
A residue theorem states that
\begin{equation}
\oint_C F(z)dz=2 \pi j \sum_k Res[F(z),p_k]
\label{eq3}
\end{equation}
Applying (\ref{eq3}) to (\ref{eq1}), 
it is straightforward to see that
\begin{equation}
1 + 1 = \pi
\label{eq4}
\end{equation}

And here is an included image (png and pdf formats are allowed).

\begin{center}
  \includegraphics[width=6cm]{figseven.png}
\end{center}
\subsection{References}

References should be numbered in order of appearance, 
for example \cite{ES1}, \cite{ES2}, and \cite{ES3}. 
You \emph{can} use \texttt{bibtex} to prepare references,
or do it by hand if there are very few.

%
\bibliographystyle{IEEEtran}
\begin{thebibliography}{10}
\bibitem[1]{ES1} Smith, J. O. and Abel, J. S., 
``Bark and {ERB} Bilinear Transforms'', 
IEEE Trans. Speech and Audio Proc., 7(6):697--708, 1999.  
\bibitem[2]{ES2} Lee, K.-F., Automatic Speech Recognition: 
The Development of the 
SPHINX SYSTEM, Kluwer Academic Publishers, Boston, 1989.
\bibitem[3]{ES3} Rudnicky, A. I., Polifroni, Thayer, E. H.,
 and Brennan, R. A.  
"Interactive problem solving with speech", J. Acoust. Soc. Amer., 
Vol. 84, 1988, p S213(A).
\end{thebibliography}
\end{document}
